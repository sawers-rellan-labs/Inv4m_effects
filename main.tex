\documentclass[9pt,twocolumn,twoside]{rilabRxiv}
% Use the documentclass option 'lineno' to view line numbers
\setlength{\marginparwidth}{2cm}
\usepackage[textsize=tiny,colorinlistoftodos]{todonotes}
\definecolor{cornflowerblue}{rgb}{0.39, 0.58, 0.93}

% Custom macros from main_20251203.tex
\usepackage{xspace}
\usepackage{booktabs}
\usepackage{multirow}

% Set supplement numbers to S and start counting newly
\newcommand{\beginsupplement}{%
  \setcounter{table}{0}
  \renewcommand{\thetable}{S\arabic{table}}%
  \setcounter{figure}{0}
  \renewcommand{\thefigure}{S\arabic{figure}}%
}

% Custom commands from source
\newcommand{\mex}{\textit{mexicana}\xspace}
\newcommand{\invfour}{\textit{Inv4m}\xspace}
\newcommand{\fdrgt} {$\textrm{\textit{FDR}} > 0.05$}
\newcommand{\fdreq} {$\textrm{\textit{FDR}} = 0.05$}
\newcommand{\fdrls} {$\textrm{\textit{FDR}} < 0.05$}
\newcommand{\parv}{\textit{parviglumis}\xspace}
\newcommand{\jmjii}{\textit{jmj2}\xspace}
\newcommand{\jmjiv}{\textit{jmj4}\xspace}
\newcommand{\jmjvi}{\textit{jmj6}\xspace}
\newcommand{\jmjix}{\textit{jmj9}\xspace}
\newcommand{\arabidopis}{\textit{Arabidopsis}\xspace}

\usepackage{hyperref}

\newcolumntype{b}{X}
\newcolumntype{s}{>{\hsize=.5\hsize}X}

\title{Introgression of Mexican highland chromosomal inversion \textit{Inv4m} into temperate 
maize accelerates flowering, promotes growth, and modulates a cell 
proliferation gene network}


\author[$1$,$2$,*]{Fausto Rodríguez-Zapata}
\author[$1$,$2$]{Nirwan Tandukar}
\author[$1$]{Hannah Pil}
\author[$1$]{Carolina Escalona-Weldt}
\author[$1$]{Zehta Glover}
\author[$1$]{Lauren Insko}
\author[$1$, $5$]{Alejandro Aragón-Raygoza}
\author[$4$]{Melanie Perryman}
\author[$4$]{Sergio Pérez-Limón}
% Li, Meng
\author[$1$,$5$]{Josh Strable}
\author[$6$]{Daniel Runcie}
\author[$4$]{Ruairidh Sawers}
\author[$1$,*]{Rubén Rellán-Álvarez}
\affil[$1$,*]{Department of Molecular and Structural Biochemistry, N.C. Plant Sciences Initiative, North Carolina State University, Raleigh, NC, USA.}
\affil[$2$]{Genetics and Genomics Program, North Carolina State University, Raleigh, NC, USA}
\affil[$3$]{United States Department of Agriculture, Agricultural Research Service, Plant Science Research Unit, Raleigh, NC 27695}
\affil[$4$]{Department of Plant Science, Pennsylvania State University, University Park, PA, USA}
\affil[$5$]{Department of Genetics, Development, and Cell Biology, Iowa State University, USA}
\affil[$6$]{Department of Plant Sciences, University of California, Davis, CA, USA}



\keywords{ Local Adaptation, Highland Maize, Teosinte Mexicana Introgression, Chromosomal Inversion}

\runningtitle{The Role of \textit{Invm4} in adaptation to low phosphorus availability} % For use in the footer

%% For the footnote.
%% Give the last name of the first author if only one author;
\runningauthor{Rodríguez-Zapata}
%% last names of both authors if there are two authors;
% \runningauthor{FirstAuthorLastname and SecondAuthorLastname}
%% last name of the first author followed by et al, if more than two authors.
\runningauthor{Rodríguez-Zapata \textit{et al.}}


%%% Abstract %%%%%%%%%%%%%%%%%%
\begin{abstract}
Maize adaptation to temperate zones may have been facilitated by introgression 
from highland wild relatives. We investigated the effects of \invfour, a 15 Mb 
chromosomal inversion from \textit{Zea mays ssp. mexicana}, by introgressing 
it into temperate maize (B73) through eight generations of backcrossing. Field 
trials under contrasting phosphorus treatments revealed that \invfour 
significantly accelerates flowering time and increases plant height, 
independent of phosphorus availability. Transcriptomic analysis identified 
perturbations in a cell proliferation gene expression network, including 
downregulation of JUMONJI methyltransferases that show higher copy numbers in 
modern maize compared to highland teosinte. Phenotypic measurements confirmed 
that \invfour plants exhibit elongated shoot apical meristems. While classical 
phosphorus starvation response genes were detected within the inversion, 
\invfour showed no detectable effect on phosphorus starvation responses. These 
findings illuminate the genetic mechanisms underlying \invfour's contribution 
to local adaptation and identify candidate genes driving its phenotypic effects.
\end{abstract}
%%%%%%%%%%%%%%%%%%%%%%%%%%

\setboolean{displaycopyright}{true}

\begin{document}

\maketitle
\thispagestyle{firststyle}
\correspondingauthoraffiliation{
Dept. of Plant Sciences, University of California, Davis, CA, USA
E-mail: corresponding@ucdavis.edu}
\vspace{-11pt}%

\setboolean{displaylineno}{true}
\ifthenelse{\boolean{displaylineno}}{\linenumbers}{}

\section{Introduction}
\lettrine[lines=2]{\color{color2}M}{aize} might have preadapted to temperate 
zones through introgression from its highland wild relative, 
\textit{Zea mays ssp. mexicana} (shorthand \mex) \cite{yang2023}. Maize was originally domesticated in the tropical lowlands of Mexico. Before its expansion into temperate regions, maize was introduced to the Mexican highlands 
and the Southwestern United States, where sympatry with highland teosinte 
likely facilitated the introgression of adaptive alleles from \mex.

However, not all highland-adaptive loci are present in temperate maize. 
Highland-associated chromosomal inversions, such as \invfour and 
\textit{Inv9f}, are prevalent in highland teosinte populations 
\cite{pyhajarvi2013} and traditional Mexican maize varieties (TVs) 
\cite{crow2020,gonzalez-segovia2019-jy} but are rare in temperate maize. 
Chromosomal inversions can contribute to local adaptation by preserving locally 
adapted alleles across multiple loci and reducing recombination within the 
inversion \cite{kirkpatrick2006b}. Genotyping of teosinte populations using 
the Maize 50K chip revealed that \invfour spans 13 Mb and is predominantly 
found in \mex populations \cite{pyhajarvi2013}. In Mexican TVs, variation in 
\invfour is associated with elevation and flowering time 
\cite{romero_navarro2017-cn}. Additionally, \invfour shows reduced genetic 
diversity, a clinal relationship with elevation, and is nearly fixed in 
locations above 2500 m.a.s.l. \cite{crow2020}. The inversion exhibits 
suppressed recombination, as confirmed in a biparental cross 
\cite{gonzalez-segovia2019-jy}.

\invfour demonstrates classic patterns of gene-by-environment interactions 
indicative of local adaptation. Plants carrying the \invfour-highland allele 
exhibit delayed flowering at low elevations and earlier flowering at high 
elevations \cite{crow2020}. The highland haplotype of \invfour was introgressed 
from \textit{Zea mays ssp. mexicana} 
\cite{pyhajarvi2013,calfee2021-mr,hufford2013-gs}, a wild maize relative 
native to the Mexican highlands.

Despite strong evidence linking \invfour to local adaptation, the physiological 
processes and environmental factors underlying its adaptive role remain unclear. 
Furthermore, the specific genes within \invfour that drive local adaptation are 
largely unidentified. Previous research has shown that \invfour-highland 
upregulates photosynthesis genes in response to cold at the seedling stage 
\cite{crow2020} and is associated with earlier flowering in the Mexican 
highlands, which likely enhances fitness in environments with limited 
growth-degree accumulation throughout the year \cite{romero_navarro2017-cn}. 
However, cold is not the only limiting factor for plant growth in the highlands.

Volcanic soils (Andosols), which dominate the Mexican highlands, present an 
additional constraint. Approximately 95\% of natural Andosol profiles in Mexico 
are found above 2000 m.a.s.l. \cite{paz-pellat2018,inegi2013}. These soils are 
characterized by high phosphorus retention \cite{krasilnikov2013}, which leads 
to low phosphorus availability for plant uptake \cite{galvan-tejada2014}. 
MICH21, one of the Mexican highland maize accessions analyzed by 
\cite{crow2020}, originates from the Purépecha Plateau, where Andosols and 
phosphorus-efficient TVs are common 
\cite{paz-pellat2018,galvan-tejada2014,bayuelo-jimenez2011,bayuelo-jimenez2014}. 
\invfour may contribute to adaptation in the highlands by carrying alleles that 
enhance the phosphorus starvation response (PSR). For example, the phosphate 
transporter gene \textit{ZmPho1;2a}, located within \invfour, is a strong 
candidate for adaptation to low phosphorus availability 
\cite{salazar-vidal2016-rl}.

In this study, we aimed to understand the physiological and molecular effects 
of \invfour and to identify candidate genes within the inversion that could 
elucidate its adaptive role. Specifically, we tested whether \invfour-highland 
contributes to adaptation to low phosphorus availability. To achieve this, we 
backcrossed MICH21, a Mexican highland TV carrying \invfour, into the B73 
genetic background for eight generations, generating Near Isogenic Lines (NILs) 
\cite{crow2020}. These NILs were grown under temperate field conditions with 
two phosphorus treatments to evaluate flowering time, height, and 
transcriptomic responses.

We observed that \invfour significantly reduces flowering time and increases 
plant height, independent of phosphorus levels. We identified a cluster of 
JUMONJI methyltransferases, which have higher copy numbers in modern maize 
compared to highland teosinte and TVs, as potential contributors to flowering 
time differences. Additionally, we found a group of cell cycle-related genes 
that may underlie height differences. While classical PSR genes were identified 
within the inversion, \invfour had no detectable effect on the phosphorus 
starvation response. These findings provide insights into the genetic 
mechanisms underlying \invfour's contribution to local adaptation and highlight 
potential candidate genes driving its effects.

\section{Results}

\subsection{\invfour is a 15 Mb inversion delimited by breakpoint segments 
overlapping chromosomal knob repeats}

Complete genome sequences of selected Mexican Highland traditional varieties 
allowed us to delimit the recombination breakpoints of \invfour by using an 
Anchorwave \cite{song2022} alignment of chromosome 4. With this aim, we aligned 
both PT, a close relative to the \invfour donor parent Mi21, as a 
representative genome for the Mexican highlands, and TIL18 representing the 
highland teosinte \mex, to B73, the recurrent parent used for \invfour 
introgression (Fig~\ref{fig::design}A, Supporting Table~\ref{tab:breakpoints}).

In each pairwise comparison, we found that the anchor-defined \invfour 
alignment is between the reference B73 plus strand and the highland genome 
minus strand, as expected. This alignment allowed us to narrow down \invfour 
location between genes \textit{Zm00001eb190470} and \textit{Zm00001eb194800}, 
spanning 15.2 Mb and 432 annotated genes in the B73 v5 genome. This region 
corresponded to a 13.4 Mb inversion in the PT genome, bounded by genes 
\textit{Zm00109aa017629} and \textit{Zm00109aa018009}; and, also a 13.2 Mb 
inversion in TIL18, extending from genes \textit{Zx00002aa015554} to 
\textit{Zx00002aa015905}.

However, the \invfour section was bordered by unaligned segments rather than by 
reference orientation (plus/plus) alignments. These unaligned segments 
contained no annotated genes, except for \textit{Zm00001eb190490} in the 
upstream breakpoint segment of B73. To explore the nature of the sequences 
spanning the \invfour breakpoints, we made a local similarity search using LAST 
\cite{kielbasa2011} inside these \invfour bounding unaligned segments. We 
detected typical tandem array patterns in the dotplots (blue and red rectangles 
in Fig~\ref{fig::design}B) and we took a closer look at the available repeat 
annotation of B73.

The MaizeGDB annotation of B73 shows that the upstream \invfour breakpoint 
segment contained 438 \textit{knob180} and 29 \textit{TR-1} repeats making up 
$69\%$ and $5\%$ of the 636 annotated repeats in this 323 kb section. These 
chromosomal knob 180 bp and TR-1 repeat annotations matched the observed tandem 
array patterns observed in the dotplots Fig~\ref{fig::design}B. Then we 
searched the two classes of knob repeats in TIL18 and PT, which have no public 
repetitive element annotation to date. For comparison purposes, we did a BLAST+ 
\cite{camacho2009} similarity search of the three chromosomes 4 using a 
selected sequence of each repeat class as subjects (see methods).

This allowed us not only to detect both kinds of knob repeats in the breakpoint 
segments but also inside the inversion and the rest of the chromosome. We found 
in the long arm of chromosome 4 repeat arrays corresponding to previously 
reported heterochromatic knobs, the ones interspersed in the \invfour 
breakpoints, and an internal knob repeat array inside \invfour 
Fig~\ref{fig::design}A. Chromosomal knob repeats were the most abundant kind of 
repeats within the \invfour breakpoint sections Table \ref{tab:breakpoints} and 
the \invfour proper.

There is a notable difference in the heterochromatic knobs between the long 
arms of the analyzed chromosomes. TIL18 has a 30 Mb region of mixed repeats 
likely corresponding to the 4L knob reported in \mex 
\cite{bilinski2018,albert2010}, PT is knobless, and B73 carries a 4L knob 
consisting of \textit{TR-1} repeats \cite{ghaffari2013,albert2010} 
(Fig~\ref{fig::design}A).

Although the downstream breakpoint had a lower content of knob repeats with 37 
\textit{knob180} and no annotated \textit{TR-1}'s, those made up $46\%$ out of 
the 81 annotated repeats in 80 kb. In each genome, the breakpoint segments on 
each side of the inversion showed asymmetries in size, knob repeat number, 
presence of knob inverted repeat arrays, and closeness of the knob repeats to 
the \invfour anchor gene (Fig~\ref{fig::design}B).

In particular, in B73 the upstream breakpoint segment is larger (323 kb vs 89 
kb), has a greater number of repeats (352 vs 26), presence of inverted knob 
repeat arrays (blue and red rectangles in Fig~\ref{fig::design} B Upstream), 
and some knob repeats close to the inversion, while the downstream breakpoint 
segment is smaller, has a lower number of repeats, no inverted repeats of knob 
sequences (just red rectangles in Fig~\ref{fig::design}B, Downstream), and the 
few knob repeats present are closer to the downstream external anchor gene 
rather than to \invfour. We can observe these four patterns in the opposite 
orientation in PT and TIL18 relative to B73.
\clearpage

\begin{figure*}[ht!]
% (fausto) remove include graphics before submission
\centering
\includegraphics[width=0.9\linewidth]{figs/design.png}
\caption[\textit{\invfour} Delimitation, Introgression Breeding Design and Population Genotypes.]{
\textbf{Chromosomal Inversion \invfour: Delimitation, Introgression Breeding Design, Field Experimental Setup and Population Genotype}.
% \textbf{(A)} Chromosome 4 macrosynteny among highland and lowland \textit{Zea} genomes illustrating \invfour. The highland Mexican traditional variety Palomero Toluqueño (PT) shows the same gene order as the teosinte \mex inbred TIL18. This highland karyotype (purple) has the \invfour inversion relative to the reference genome inbred B73 (\textit{spp. mays}, yellow). Synteny plot created with MCscan\cite{tang2024}, genes from the ortholog groups defining the endpoints of the inversion are depicted below each ideogram.
\textbf{(A)} \invfour breakpoints overlap clusters of full 180 bp knob repeats with interspersed partial \textit{TR-1}s (normalized match score $<50$). Each point represents a BLAST+ \cite{camacho2009} hit with a selected sequence of 180 bp knob or  \textit{TR-1} repeats. 
 Around the 230 Mb mark regions corresponding to reported 4L cytogenetic knobs can be observed in \mex and B73 while absent in PT.
\textbf{(B)} Sequence self-similarity dotplots, using LAST \cite{kielbasa2011}, for the inversion breakpoint segments as defined in Table~\ref{tab:breakpoints}.
Breakpoints are upstream and downstream of \invfour in each genome coordinate system as depicted in (A).
Forward matches in red, reverse in blue. 
Rectangles indicate tandem repeats. 
\textbf{(C)} Near Isogenic Line population breeding scheme. The \invfour from Mi21, a highland traditional variety closely related to PT, was introgressed into a temperate lowland genetic background using B73 as recurrent parent.
\textbf{(F)} NIL genotypes at BC$_6$S$_2  $, of plants used in RNA and lipid extractions, $n=13$, B73 NAM5 coordinates. 
\textbf{(G)} Zoom into \invfour introgression. Plants are sorted by genotype at the \invfour tagging SNP PZE04175660223 (181.6 Mb, downwards pointing triangle) then by field row number. 
}
\label{fig::design}
\end{figure*}

\clearpage

\subsection{Divergent highland introgression is mostly confined to a 39 Mb 
segment spanning \invfour}

To study the effects of \invfour we repeatedly backcrossed Mi21, a Mexican 
highland landrace containing \invfour into B73 using a marker inside the 
inversion. We then used RNA-Seq data to genotype the resulting 
$\text{BC}_6\text{S}_2$ lines. The distribution of the 19861 QC-filtered SNPs 
is heavily biased towards chromosome 4. Almost half of the variants, 8892 
(44\%), are on chromosome 4, with the other 9 chromosomes having an average of 
1219 SNPs (6\%) each (Supporting Fig~\ref{fig::SNPdistro}A).

The genotypes in the matrix of 19861 SNPs $\times$ 13 individuals are 63.5\% 
homozygous for B73, 8.96\% heterozygous, and 27.3\% homozygous for Mi21, with 
0.16\% missing data. We converted this genotype matrix to identical genotype 
run \cite{layer2016} length in base pairs to estimate the proportion of the 
plant genome introgressed from Mi21 (Fig~\ref{fig::design}F and G).

By genotype run length, the plants are, on average, 85.6\% homozygous for B73 
(1821 Mb), 8.61\% heterozygous (183 Mb), and 5.5\% homozygous for Mi21 (117 
Mb), with 0.34 \% missing data (7.32 Mb) (Supporting Fig~\ref{fig::SNPdistro}D). 
As the genotypes come exclusively from variant sites, the matrix has no markers 
fixed for the reference B73 allele. The NIL genomes are thus represented by a 
mosaic of 3 types of sections. First, sections with fixed highland 
introgression, where the alternate Mi21 allele is fixed. Second, sections of 
divergent highland introgression consisting of markers significantly correlated 
with the \invfour tagging SNP PZE04175660223. And third, sections with random 
introgressions uncorrelated with \invfour.

The fixed highland introgression sections are made up of 126 SNPs marking 51 Mb 
(2.4\% of the genome) in genotype run length (average per plant). By manual 
inspection we defined a region of divergent highland introgression between 
positions 157012149 and 195900523 in B73 v5 coordinates, bounding 39 Mb of 
shared introgression among \invfour carrying lines. With our selection for the 
highland allele of PZE04175660223 we have produced plants with the introgression 
of the target \invfour inversion surrounded by 24 Mb of lingering unintended 
linkage drag (Fig~\ref{fig::design}F and G). Conversely, we have produced 
inversion-free lines in the CTRL group by selecting for the B73 allele.

In more detail, there are 7683 SNPs significantly correlated with \invfour 
(Pearson correlation \textit{t-test}, \fdrls) constituting divergent highland 
introgressions throughout the genome but mostly in \invfour and flanking regions 
(Supporting Fig~\ref{fig::SNPdistro}A). From these 7683, 7322 SNPs ($95.3\%$) 
cover 35 Mb and 271 expressed genes inside the 39 Mb shared introgressed region 
that includes \invfour. The remaining 361 divergently introgressed SNPs in the 
genomic background were dispersed throughout 10.4 Mb of sequence covering 52 
expressed genes.

Notably, the majority of these markers, 207 SNPs from 14 expressed genes, are 
negatively correlated with PZE04175660223 and are located just upstream of the 
shared introgression near \invfour over a 2.7 Mb stretch, from position 
154283637 to 15698557 (Fig~\ref{fig::design}G and Supporting 
Fig~\ref{fig::SNPdistro}C). In this segment, individuals selected for \invfour 
are homozygous for the B73 allele at all positions, while individuals selected 
for the standard karyotype are frequently heterozygous.

In summary, each plant has an average of 117 Mb of homozygous Mi21 genome per 
line, 51 Mb of fixed highland introgression, 35 Mb of divergent highland 
introgression colocalized and matching the selection of \invfour, and 10 Mb of 
other dispersed divergent highland introgressions.


  \begin{figure*}[!ht]
  \includegraphics[width=0.95\linewidth]{figs/effects.png}
  \centering
  \caption{\textbf{Effect of \invfour on field phenotypes and shoot apical meristem architecture.}
  \textbf{(A)} Spatially-corrected phenotype values (BLUEs from SpATS mixed models) for plant height (PH, cm), days to anthesis (DTA), days to silking (DTS), and harvest index (HI) comparing CTRL (yellow) and \invfour (purple) genotypes. Phosphorus treatment was modeled as a covariate and data are pooled across treatments.
  \textbf{(C)} Representative differential interference contrast micrographs of cleared vegetative shoot apical meristems from two-week-old seedlings grown in controlled environment chambers. Yellow lines indicate height and width measurements used for quantification. Scale bars: 100 $\mu$m.
  \textbf{(D)} Quantification of SAM dimensions: absolute height ($h$), radius ($r$), height-to-radius ratio ($h/r$), and shape factor ($h/r^2$). \invfour plants exhibit significantly taller meristems with elongated shape.
  Significance: $^{*}p < 0.05$, $^{****}p < 0.0001$ (two-sample \textit{t}-test, FDR-adjusted).}
  \label{fig:phenotypes}
  \end{figure*}

\subsection{Inv4m accelerates flowering and promotes growth}

Field trials demonstrated that \invfour significantly affects both flowering 
time and plant architecture. Plants carrying the highland inversion allele 
flowered earlier and grew taller than control lines, with these effects 
consistent across both phosphorus treatments (Fig~\ref{fig:phenotypes}).

\begin{figure*}[!ht]
\centering
\includegraphics[width=0.8\linewidth]{figs/RNAseq.png}
\caption{\textbf{Global and local transcriptomic effects of \invfour across leaf stages.}
\textbf{(A)} Leaf sampling strategy. Four developmental leaf stages (1 to 4) were sampled across four replicates for both \invfour and CTRL genotypes. 
\textbf{(B)} Multidimensional Scaling Analysis of the leaf transcriptome. Samples are colored by genotype (CTRL: yellow; \invfour: purple) and shaped by leaf stage, showing clear separation along dimension 3. 
\textbf{(C)} Volcano plot of \invfour-associated DEGs. Labeled genes highlight significant perturbations, including the downregulation of multiple JUMONJI methyltransferases within the inversion. 
\textbf{(D)} Global Manhattan plot showing Differentially Expressed Genes (DEGs) associated with leaf stage across the ten maize chromosomes. The floral integrator \textit{zmm4} is a highly significant DEG. 
\textbf{(E)} Global Manhattan plot for DEGs associated with the \invfour genotype effect. A prominent peak of differential expression is localized to the introgressed region on chromosome 4, containing the \jmjii, \jmjvi, and \jmjix cluster. 
\textbf{(G)} Haplotype map of the introgressed region on chromosome 4 for representative NIL lines. The blue triangle indicates the tagging SNP PZE04175660223 used for selection. 
\textbf{(H)} Zoome into the 150-200 Mb region of chromosome 4, showing  the effect of \invfour introgression in gene expression. 
\textbf{(I)} Expressoin effect of Inv4,across chromosome 4. Moving average lines (green for upregulated, red for downregulated) illustrate the coordinated transcriptomic shift within the \invfour region, specifically the downregulation of the \jmjii cluster.
}
\label{fig:transcriptome}
\end{figure*}


\begin{table*}[ht!]
\caption{Distribution of \invfour DEGs in the Maize Genome}
\label{tab:DEGs_distro}
\resizebox{\textwidth}{!}{%
\begin{tabular}{@{}lrrrrrr@{}}
 &
  \multicolumn{1}{c}{} &
  \multicolumn{1}{c}{} &
  \multicolumn{1}{c}{} &
  \multicolumn{1}{c}{\textbf{}} &
  \multicolumn{1}{c}{\textbf{DEGs}} &
  \multicolumn{1}{l}{\textbf{High-conf DEGs}} \\
\textbf{Location of genes} &
  \multicolumn{2}{c}{\textbf{\begin{tabular}[c]{@{}c@{}}Total \\ Annotated\end{tabular}}} &
  \multicolumn{1}{c}{\textbf{\begin{tabular}[c]{@{}c@{}}Total\\  Expressed\end{tabular}}} &
  \multicolumn{1}{c}{\textbf{\begin{tabular}[c]{@{}c@{}}Fraction \\ Expressed\end{tabular}}} &
  \multicolumn{1}{c}{$\textit{FDR} < 0.05$} &
  \multicolumn{1}{c}{\begin{tabular}[c]{@{}c@{}}$\textit{FDR} < 0.05$ \\ $|log_2 FC| > 1.5$\end{tabular}} \\ \midrule
Chromosomes 1-10 &
  \multicolumn{2}{r}{43459} &
  24011 &
  0.55 &
  465 &
  155 \\
Outside Introgression &
  \multicolumn{2}{r}{42360} &
  23403 &
  0.55 &
  193 &
  39 \\
Shared Introgression &
  \multicolumn{2}{r}{1099} &
  608 &
  0.55 &
  272 &
  116 \\
\hspace*{2em}\invfour &
  \multicolumn{2}{r}{432} &
  253 &
  0.59 &
  114 &
  40 \\
\hspace*{2em}Flanking Introgression &
  \multicolumn{2}{r}{667} &
  355 &
  0.53 &
  158 &
  76 \\ \midrule
 &
  \multicolumn{1}{l}{} &
  \multicolumn{1}{l}{} &
  \multicolumn{1}{l}{} &
  \multicolumn{1}{l}{} &
  \multicolumn{1}{l}{} &
  \multicolumn{1}{l}{} \\
 &
  \multicolumn{3}{c}{\textbf{DEGs Enrichment}} &
  \multicolumn{3}{c}{\textbf{High-conf DEGs Enrichment}} \\ \cline{2-7}
\textbf{Comparison} &
  \multicolumn{1}{c}{\textbf{\begin{tabular}[c]{@{}c@{}}Odds\\ Ratio\end{tabular}}} &
  \multicolumn{1}{c}{\textbf{\begin{tabular}[c]{@{}c@{}}Fisher's \\ Exact test\\ \textit{p-value}\end{tabular}}} &
  \multicolumn{1}{c}{\textbf{\begin{tabular}[c]{@{}c@{}}Confidence\\ Interval\end{tabular}}} &
  \textbf{\begin{tabular}[c]{@{}r@{}}Odds\\ Ratio\end{tabular}} &
  \multicolumn{1}{c}{\textbf{\begin{tabular}[c]{@{}c@{}}Fisher's\\ Exact test\\ \textit{p-value}\end{tabular}}} &
  \multicolumn{1}{c}{\textbf{\begin{tabular}[c]{@{}c@{}}Confidence\\ Interval\end{tabular}}} \\ \midrule
Shared introgression vs Outside &
  97.1 &
  $<10^{-200}$ &
  {[}70,135{]} &
  141 &
  $<10^{-100}$ &
  {[}95,215{]} \\
Flanking vs Outside &
  96.5 &
  $<10^{-150}$ &
  {[}68,135{]} &
  163 &
  $<10^{-80}$ &
  {[}100,275{]} \\
\textbf{\invfour vs Outside} &
  \textbf{98.7} &
  $<10^{-100}$ &
  {[}65,150{]} &
  \textbf{112} &
  $<10^{-45}$ &
  {[}65,200{]} \\
\invfour vs Flanking &
  1.02 &
  0.95 &
  {[}0.7,1.5{]} &
  0.69 &
  0.25 &
  {[}0.4,1.2{]} \\ \bottomrule
\end{tabular}%
}
\end{table*}
\clearpage


\subsection{Despite strong transcriptional effects, \invfour differential 
expression lacks functional enrichment}

The leaf gene expression and lipid profile reveal distinct molecular and 
metabolic responses to leaf stage, phosphorus availability, and \invfour 
genotype (Fig~\ref{fig::volcano}). Increased leaf stage, which correlates with 
aging and senescence, leads to the downregulation of genes involved in floral 
meristem determinacy (GO biological process enrichment, Fisher's exact test 
\textit{FDR} $=1.28\times 10^{-5}$). This set was constituted by the MADS box 
transcription factors \textit{zmm4}, \textit{mads45}, \textit{zmm15}, 
\textit{mads67}; \textit{zap1} (a maize \textit{APETALA1} homolog); and the TNF 
receptor-associated factor, \textit{traf7}.

Conversely, samples from later leaf stages show upregulation in a set of genes 
enriched in aging (Fisher's exact test, \textit{FDR} $=0.002$) that includes 
the diaminopimelate aminotransferase \textit{dapat3} and the \textit{SAG12} 
cysteine protease homolog \textit{ccp16}. The KEGG pathway annotation shows 
that genes encoding photosynthetic antenna proteins (Fisher's exact test, 
\textit{FDR} $=0.015$) are significantly downregulated in phosphorus deficiency 
(\textit{lhcb10}) and increased leaf stage (\textit{lhcb1}, \textit{lhcb10}) 
Fig~\ref{fig::volcano} B.

Downregulated genes under phosphorus starvation do not show enrichment in 
specific biological processes. However, the most significant PSR gene was 
\textit{peamt2}, a phosphoethanolamine N-methyltransferase, which is involved 
in phospholipid biosynthesis, congruent with the reduction in phospholipid 
synthesis under phosphorus-limiting conditions. On the other hand, plants 
growing under phosphorus deficiency showed an evident upregulation of classical 
phosphorus starvation response genes.

Although not included in the GO annotation of protein-coding genes, the most 
prominent response is that of \textit{pilncr1}, a long noncoding RNA spanning 
\textit{mir397}, a known master regulator of PSR. The upregulated protein-coding 
genes showed enrichment in cellular response to phosphate starvation (Fisher's 
exact test, \textit{FDR} $=9.07 \times 10^{-11}$), including \textit{SPX} 
family transcription factors, and phosphate transporters such as \textit{phos1}, 
which facilitate phosphate uptake and redistribution \textit{pht1} \textit{pht7}, 
and purple acid phosphatases \textit{pap1} \textit{pap14} that increase 
phosphorus remobilization.

Over-representation analysis of KEGG metabolic pathways showed that phosphorus 
starvation upregulates genes involved in glycerophospholipid metabolism 
(glycerophosphodiester phosphodiesterases \textit{gpx1}, \textit{gpx3}, 
monogalactosyldiacylglycerol synthase \textit{mgd2}), phenylpropanoid and 
flavonoid biosynthesis, chitin degradation, ABC transporters, and 
brassinosteroid biosynthesis (\textit{brc2}). Conversely, genes involved in 
galactolipid biosynthesis, which are crucial for maintaining photosynthetic 
membranes are significantly downregulated.

\clearpage

\subsection{\invfour plants show perturbations of a cell proliferation gene 
expression network in leaves and have elongated shoot apical meristems}

A gene co-expression network analysis (Panel A) was performed to investigate the 
trans-regulatory effects associated with the Inv4m chromosomal inversion in 
maize leaves. This network comprised approximately 24 genes and 28 co-expression 
edges, filtered to represent interactions between genes within the 15 Mb Inv4m 
inversion on chromosome 4 and genes located elsewhere in the genome. Of these, 
12 genes were located within the inversion, and 12 genes were located outside 
the inversion. We observed that 10 genes within the network were upregulated 
and 5 were downregulated in Inv4m plants.

Gene ontology (GO) enrichment analysis (Panel B) of this Inv4m 
trans-coexpression network revealed an over-representation of terms related to 
cell population proliferation and regulation of flower development (Fisher's 
exact test p-values here). Within the enriched group for cell population 
proliferation, we observed that the expression of mrlk (meristematic 
receptor-like kinase), PCNA2 (encoding a proliferative cell nuclear antigen 2 
protein, part of a clamp complex increasing processivity of DNA polymerase 
delta), and MCM5 (encoding a minichromosome maintenance 5 protein, a helicase) 
were all perturbed in Inv4m plants.

We specifically noted the perturbation of zcn26, an FT homolog (a gene known 
for its role in flowering time) located outside the inversion. Furthermore, we 
identified disruptions in genes involved in protein methylation, specifically 
\textit{jmj2} and \textit{jmj21} (JMJ domain-containing proteins, shown in the 
network), suggesting a potential role for epigenetic regulation.

To determine whether these observed gene expression changes had phenotypic 
consequences, we measured the shape and height of the vegetative meristems in 
control and Inv4m individuals. We found that Inv4m plants have shoots with a 
significant ($p<0.05$, one-tailed \textit{t-test}, 
$H_a: \mu_{\textit{Inv4}} > \mu_{\textit{CTRL}}$) 9\% increase in SAM height 
(9.65 $\mu$m, 0.37 $\mu$m lower bound with 95\% confidence), a 9\% increase in 
shape coefficient (2.49 $\text{nm}^{-1}$, 0.59 $\text{nm}^{-1}$), indicating an 
elongated meristem (Panel C).

  \begin{figure*}[!ht]
  \centering
  \includegraphics[width= 0.85\linewidth]{figs/DEG_network.png}
  \caption{\textbf{Trans coexpression network connecting \invfour DEGs to genome-wide targets.}
  \textbf{(A)} Minimum spanning tree visualization of co-expression relationships between DEGs within the \invfour introgression (purple) and target genes located elsewhere in the genome (yellow). Node fill indicates direction of differential expression: filled nodes are upregulated and open nodes are downregulated in \invfour plants. Edge polarity reflects the sign of mutual information: arrows indicate positive coexpression and bars indicate negative coexpression.
  \textit{Left:} Reference edges present in the MaizeNetome database, representing known regulatory relationships. The network includes key cell cycle genes (\textit{pcna2}, \textit{map4k8}), transcription factors (\textit{ocl2}), and the \jmjii cluster.
  \textit{Right:} Novel edges specific to the \invfour network, showing the \jmjiv neighborhood subgraph. This includes upregulated genes involved in cell proliferation (\textit{sec6}, \textit{kan1}) and downregulated JUMONJI methyltransferases.
  \textbf{(B)} Gene Ontology enrichment analysis of the trans coexpression network, stratified by gene location (flanking introgression vs. \invfour proper) and regulation direction. Bubble size indicates gene count; color indicates enrichment significance ($-\log_{10}(p)$). Gene labels identify representative members of enriched terms. Notable enrichments include negative regulation of gene expression (epigenetic) and positive regulation of growth among \invfour-downregulated genes, consistent with the observed meristem phenotype.}
  \label{fig::DEGnetwork}
  \end{figure*}

\begin{table*}[!ht]
\caption{\textbf{Effect of \invfour on Flowering Time and Plant Height Gene 
Candidates}}
\label{tab:FT_PH_candidates}
\resizebox{\textwidth}{!}{%
\begin{tabular}{lllllllll}
\hline
\rowcolor[HTML]{FFFFFF} 
\textbf{Phenotype} &
  \textbf{Gene} &
  \textbf{Label} &
  \textbf{Description} &
  \textbf{\begin{tabular}[c]{@{}l@{}}$\bf{log_2}$\\(FC)\end{tabular}} &
  \textbf{\begin{tabular}[c]{@{}l@{}}$\bf{-log_{10}}$\\ 
(\textit{FDR})\end{tabular}} &
  \textbf{Location} &
  \textbf{\begin{tabular}[c]{@{}l@{}} GWAS \\ 
\textit{p-value}\end{tabular}} &
  \textbf{ref} \\ \midrule
\rowcolor[HTML]{EFEFEF}
\cellcolor[HTML]{EFEFEF} &
  \textit{Zm00001eb191790} &
  \textit{jmj2} &
  Lysine-specific demethylase &
  -3.49 &
  21.93 &
  \textbf{\invfour} &
  1.34E-07 &
  \cite{tibbs-cortes2024} \\
\rowcolor[HTML]{EFEFEF}
\cellcolor[HTML]{EFEFEF} &
  \textit{Zm00001eb191790} &
  \textit{jmj6} &
  Lysine-specific demethylase &
  -3.49 &
  21.93 &
  \textbf{\invfour} &
  1.34E-07 &
  \cite{tibbs-cortes2024} \\
\rowcolor[HTML]{EFEFEF}
\cellcolor[HTML]{EFEFEF} &
  \textit{Zm00001eb191790} &
  \textit{jmj9} &
  Lysine-specific demethylase &
  -3.49 &
  21.93 &
  \textbf{\invfour} &
  1.34E-07 &
  \cite{tibbs-cortes2024} \\
\rowcolor[HTML]{EFEFEF}
\cellcolor[HTML]{EFEFEF} &
  \textit{Zm00001eb191820} &
  \textit{jmj4} &
  Lysine-specific demethylase &
  -2.84 &
  19.72 &
  \textbf{\invfour} &
   &
  \cite{wang2021} \\
\rowcolor[HTML]{EFEFEF}
\cellcolor[HTML]{EFEFEF} &
  \textit{Zm00001eb060540} &
  \textit{actin-1} &
  Actin-1 &
  2.23 &
  9.93 &
  outside &
   &
  \cite{wang2021} \\
\rowcolor[HTML]{EFEFEF}
\cellcolor[HTML]{EFEFEF} &
  \textit{Zm00001eb070810} &
  \textit{} &
  \begin{tabular}[c]{@{}l@{}}YchF C-terminal domain\\ containing
protein\end{tabular} &
  4.19 &
  9.70 &
  outside &
   &
  \cite{wang2021, li2016, liu2023} \\
\rowcolor[HTML]{EFEFEF}
\cellcolor[HTML]{EFEFEF} &
  \textit{Zm00001eb190090} &
  \textit{} &
  Aspartic proteinase &
  7.18 &
  7.00 &
  flanking &
  1.14E-11 &
  \cite{peiffer2014} \\
\rowcolor[HTML]{EFEFEF}
\cellcolor[HTML]{EFEFEF} &
  \textit{Zm00001eb157030} &
  \textit{} &
  \begin{tabular}[c]{@{}l@{}}Electron transfer flavoprotein\\ subunit
alpha\end{tabular} &
  5.68 &
  4.84 &
  outside &
   &
  \cite{wang2021, li2016, liu2023} \\
\rowcolor[HTML]{EFEFEF}
\cellcolor[HTML]{EFEFEF} &
  \textit{Zm00001eb428740} &
  \textit{ocl2} &
  Homeobox-leucine zipper &
  -2.66 &
  5.46 &
  outside &
   &
  \cite{wang2021} \\
\rowcolor[HTML]{EFEFEF}
\cellcolor[HTML]{EFEFEF} &
  \textit{Zm00001eb197370} &
  \textit{abi40} &
  \begin{tabular}[c]{@{}l@{}}ABI3VP1-type\\ transcription
factor\end{tabular} &
  -2.25 &
  5.77 &
  outside &
   &
  \cite{wang2021} \\
\rowcolor[HTML]{EFEFEF}
\cellcolor[HTML]{EFEFEF} &
  \textit{Zm00001eb186670} &
  \textit{} &
  Uncharacterized protein &
  -2.83 &
  5.41 &
  flanking &
   &
  \cite{wang2021, li2016, liu2023} \\
\rowcolor[HTML]{EFEFEF}
\multirow{-11}{*}{\cellcolor[HTML]{EFEFEF}\begin{tabular}[c]{@{}l@{}}Flowering\\
Time\end{tabular}} &
  \textit{Zm00001eb191000} &
  \textit{} &
  4-coumarate--CoA ligase-like 5 &
  -3.35 &
  2.76 &
  \textbf{\invfour} &
   &
  \cite{wang2021} \\ \hline
 &
  \textit{Zm00001eb196600} &
  \textit{} &
  \begin{tabular}[c]{@{}l@{}}R3H-assoc domain\\ containing
protein\end{tabular} &
  -2.25 &
  12.49 &
  flanking &
  8.80E-11 &
  \cite{peiffer2014, liu2023} \\
 &
  \textit{Zm00001eb194380} &
  \textit{sec6} &
  Exocyst complex component &
  2.87 &
  16.53 &
  \textbf{\invfour} &
  1.42E-04 &
  \cite{peiffer2014, liu2023} \\
 &
  \textit{Zm00001eb196920} &
  \textit{} &
  Uncharacterized protein &
  -2.81 &
  7.37 &
  flanking &
  8.36E-08 &
  \cite{peiffer2014, liu2023} \\
 &
  \textit{Zm00001eb195080} &
  \textit{imk3} &
  \begin{tabular}[c]{@{}l@{}}Leucine-rich repeat receptor-like\\ protein
kinase IMK3\end{tabular} &
  -4.68 &
  5.34 &
  flanking &
  1.19E-09 &
  \cite{peiffer2014, liu2023} \\
 &
  \textit{Zm00001eb186670} &
  \textit{} &
  Uncharacterized protein &
  -2.83 &
  5.41 &
  flanking &
  7.18E-11 &
  \cite{peiffer2014, liu2023} \\
 &
  \textit{Zm00001eb194020} &
  \textit{exo} &
  \begin{tabular}[c]{@{}l@{}}Exordium / Phi-1-like \\ phosphate-induced
protein\end{tabular} &
  -2.39 &
  1.86 &
  \textbf{\invfour} &
  3.32E-12 &
  \cite{peiffer2014, liu2023} \\
\multirow{-7}{*}{\begin{tabular}[c]{@{}l@{}}Plant\\ Height\end{tabular}} &
  \textit{Zm00001eb197300} &
  \textit{} &
  \begin{tabular}[c]{@{}l@{}}Cysteine-rich RLK\\ (receptor-like protein
kinase)\end{tabular} &
  -2.62 &
  2.00 &
  flanking &
  3.04E-08 &
  \cite{peiffer2014, liu2023} \\ \hline
\end{tabular}%
}
\end{table*}

However, the variation at \invfour seems to have no major effect in the gene 
response to phosphorus. The \invfour x P interaction t-test (\textit{FDR} 
$<0.005$) shows only one gene with differential response to phosphorus depending 
on the \invfour karyotype. This gene is found in near \invfour, but outside 
previously reported limits (Fig~\ref{fig::RNAseq} C). A single gene 
\textit{aldh2}, 3 Mb upstream of the reported \invfour limit, shows a different 
response to phosphorus depending on the \invfour genotype. In phosphorus 
deficiency \textit{aldh2} overexpresses in the CTRL plants, but it is not 
responsive in the \invfour plant (Fig~\ref{fig::RNAseq} D).

\section{Materials and Methods}
\label{sec:materials:methods}

\subsection{\invfour Near Introgressed Lines, growth conditions, experimental 
design, and phenotype measurements}

To measure the effects of the \invfour in plant field phenotypes and their 
phosphorus starvation response transcriptome, we used a highland traditional 
variety carrying the Highland haplotype of \invfour corresponding to the 
inverted karyotype. The accession Michoacán 21 (referred to as Mi21), from the 
Mexican Cónico group, was obtained from the International Maize and Wheat 
Improvement Center (CIMMYT). In contrast, the reference genome of the temperate 
inbred B73, the recurrent parent for introgression, carries the lowland 
haplotype corresponding to the standard non-inverted karyotype at \invfour.

From the cross of Mi21 with B73 one F1 individual was backcrossed to B73 for 
six generations. We selected lines carrying \invfour with a diagnostic SNP 
during each cycle using a cleaved amplified polymorphic sequence (CAPS) marker. 
The marker SNP is PZE04175660223 located at position 4:181637780 in the NAM 
B73v5 \textit{Zea mays} genome assembly. Amplification of the polymorphic site 
was done with the following primer pair: 
\textit{CTGAGCAGGAGATGATGGCCACTC} and 
\textit{GGAAAGGACATAAAAGAAAGGTGCA}, and subsequently cleaved by \textit{HinfI}. 
Plants were genotyped using the CASP marker for selecting heterozygous plants 
at BC6S2 after selfing seeds of \invfour and CTRL homozygous individuals were 
selected for the field trial.

Plants were planted on May 26 2022 at the Russell E. Larson Agricultural 
Research Farm in Rock Springs, Pennsylvania (40°42'36" N 77°57'0" W, 366 
m.a.s.l.) in soil classified as a Hagerstown silt loam (fine, mixed, semiactive, 
mesic Typic Hapludalf). Experimental conditions were similar to previously 
described \cite{strock2018}. The experiment had a complete block design with 
two phosphorus (P) levels. Low-P fields (5 ppm Melich-3 Phosphorus) and high-P 
fields (36 ppm Melich-3 Phosphorus) were divided into smaller blocks. Three 
rows per block were planted with a mean stand count of 8 plants per plot, and 
the plants from the center row were selected for measurements to avoid border 
effects. Fields received fertilization based on treatment requirements. Drip 
irrigation was provided during dry periods. Each genotype was replicated four 
times within its P treatment.

\subsection{Phenotype analysis}

For stover mass growth curves, a different plant at each time point 40, 50, 60,
and harvest, 121 days after planting (DAP), was collected, dried, and weighed
for the same row. Stover dry mass data was fitted to a logistic growth model
using the R package \textit{Growthcurver} \cite{sprouffske2016}. Maximum stover
dry weight was estimated as the maximum over the four time points rather than
dry weight at harvest. Ear measurements were taken for one ear per row at
harvest.

We used a two-stage approach to estimate \invfour effects on field phenotypes
while accounting for spatial heterogeneity. In the first stage, we applied
P-spline analysis of spatial trends (SpATS) \cite{rodriguez-alvarez2018}
implemented via the \texttt{statgenHTP::fitModels()} function
\cite{millet2025}. For each phenotype $y$, SpATS fits a mixed model
that decomposes spatial variation into smooth bivariate surfaces using
two-dimensional P-splines over field row and column coordinates:

\begin{eqnarray}
\label{eq:pheno_model}
y_{ijr} = \mu + \text{Rep}_i + \text{P}_j + f(\text{row}, \text{col}) + \varepsilon_{ijr}
\end{eqnarray}

where $\mu$ is the overall mean, $\text{Rep}_i$ is the replicate effect,
$\text{P}_j$ is the phosphorus treatment effect, and
$f(\text{row}, \text{col})$ is a smooth bivariate spatial surface estimated
via penalized spline ANOVA (PSANOVA). Notably, genotype was not included in
this model, so genotypic effects are preserved in the spatially-corrected
values extracted using the \texttt{getCorrected()} function.

In the second stage, we tested for \invfour effects on these spatially-corrected
values. For each phenotype, we performed two-sample \textit{t}-tests comparing
\invfour and control genotypes, with data pooled across phosphorus treatments.
This approach tests whether \invfour affects phenotypes independent of
phosphorus availability. P-values were adjusted for multiple comparisons using
the Benjamini-Hochberg false discovery rate (FDR) method \cite{benjamini1995}.
Phenotypes with adjusted $p < 0.05$ were considered significantly affected by
\invfour.

Days to anthesis (DTA) and days to silking (DTS) were converted to growing
degree days (GDDA and GDDS, respectively) using daily temperature records from
the Rock Springs weather station with a base temperature of 10°C and an upper
threshold of 30°C.

\subsection{Tissue sampling, RNA extraction, and sequencing}

We sampled the plants at 63 DAP when we estimated them to be between v10 to v12 
developmental stages. We took tissue from he first leaf with a fully developed 
collar, or first leaf before the flag leaf, and every other leaf below for a 
total of four sampled leaves per plant. These leaves were numbered sequentially 
from 1 (most apical) to 4 (most basal). We used four replicate plants per 
combination of P treatment and \invfour genotype for a total of 64 tissue 
samples.

We took ten disc samples from the leaf tips with a tissue puncher and 
immediately froze the tissue in 1.5 mL tubes with two steel beads precooled 
with liquid nitrogen and kept in dry ice until stored at -80°C. We extracted 
total RNA with the QIAGEN RNAeasy Plant Mini Kit RNA extraction kit following 
manufacturer procedures (QIAGEN 74904), and RNA samples were quantified in 
nanodrop and sent to the NCSU Core Genomics Laboratory for sequencing. 
Following QC in Bioanalyzer, Illumina libraries were prepared and sequenced in 
a lane of Novaseq according to manufacturer recommendations.

\subsection{Plant genotyping}

We followed \cite{brouard2022} for GATK-based RNAseq genotyping of 15 plant 
samples represented by 60 leaf libraries. Briefly, Illumina short reads were 
mapped to the NAM5 Zea mays B73 genome \cite{hufford2021} using \textit{STAR} 
\cite{dobin2013}, then we marked duplicates in the resulting BAM alignments, 
split reads at intron-exon junctions and recalibrated sequence quality per leaf 
library.

At this point, we used HaploytypeCallerfor for generating gvcfs per plant 
identified by field row id ($\sim 4$ libraries per plant). We did joint sample 
genotyping afterward with \textit{genotypeGVCFs}. Then we filtered for variant 
quality (\textit{window 35, cluster, QD < 2.0, FS > 30.0, SOR > 3.0, MQ < 
40.0}) for the genotypes and $50\%$ marker completion for individuals. This 
resulted in 200000 markers with $85\%$ complete data for 13 plants. Finally, we 
used TASSEL5 K Nearest Neighbour imputing, producing a matrix of 19668 markers 
at $99.84\%$ completion. Shell scripts are available at the 
\href{https://github.com/sawers-rellan-labs/\invfourRNA}{\textit{\invfourRNA} 
github repository}

\subsection{Differential gene expression analysis}

We aligned reads to the maize Zm-B73-REFERENCE-NAM-5.0 genome using
\textit{kallisto} \cite{bray2016}. The alternative transcript alignment was
turned into counts per gene. We used \textit{voom} to calculate variance
weights according to gene expression levels and counts were converted to
$\log_2(\text{CPM})$ \cite{ritchie2015}.

We made a multivariate multiple regression for gene expression using
\textit{limma}. For the log-transformed expression $Y_{ijrs}$, from leaf
stage $s$, in plant $r$, under phosphorus treatment $i$, with genotype $j$,
we have:

\begin{eqnarray}
\begin{aligned}
Y_{ijrs} = {}& \beta_0 + \beta_1 \text{Leaf}_s + \beta_2 \text{P}_i +
\beta_3 \textit{\invfour}_j \\
& + \beta_4 [\text{Leaf} \times \text{P}]_{si} +
\beta_5 [\text{P} \times \textit{\invfour}]_{ij} \\
& + \beta_6 [\text{Leaf} \times \textit{\invfour}]_{sj} +
u_r + \varepsilon_{ijrs}
\end{aligned}
\end{eqnarray}

with plant-level random effect and residuals:
\begin{eqnarray}
u_r \sim \mathcal{N}(0, \sigma^2_u), \quad
\varepsilon_{ijrs} \sim \mathcal{N}(0, \phi_{ijrs}\sigma^2)
\end{eqnarray}

The term $u_r$ represents a plant-level random effect that accounts for
non-independence among leaves sampled from the same individual. This
approach correctly treats plants (n=4 per treatment $\times$ genotype) as
biological replicates, with leaves as correlated sub-samples within each
plant. The consensus within-plant correlation was estimated using
\textit{duplicateCorrelation} in \textit{limma} and incorporated into
model fitting.

The leaf stage ($\text{Leaf}_s$) was modeled as a centered numerical
variable ($s \in \{1,2,3,4\}$), so the intercept represents expression
at the average leaf stage. The precision weights $\phi_{ijrs}$ from the
voom transformation capture heteroskedastic measurement error across
samples.

We adjusted the p-values for the t-tests of the linear model coefficients
as false discovery rates (FDR). Genes whose effect had FDR $< 0.05$ were
deemed differentially expressed. For phosphorus treatment ($\beta_2$) and
\invfour genotype ($\beta_3$), genes with
$|\log_2(\text{FC})| > 1.5$ were considered high-confidence DEGs. For the
leaf stage effect ($\beta_1$) and interaction terms, a threshold of
$|\log_2(\text{FC})| > 0.5$ was applied, corresponding to $>2.1$-fold
total change across the leaf gradient. R scripts and expression data are
available at the
\href{https://github.com/sawers-rellan-labs/inv4m}{\invfour
github repository}.


\subsection{Gene Ontology and KEGG overrepresenattion analysis}

Once we had sets of differentially expressed genes for the three predictors 
(leaf, -P, \invfour) and two types of gene expression response (upregulated and 
downregulated), we proceeded to annotate them with gene ontology terms and KEGG 
pathways using \textit{ClusterProfiler} \cite{yu2012, zicola2024}. We started 
with the B73 NAM v5 gene ontology annotation from \cite{fattel2024} and added 
GO terms for each intermediate node in the gene ontology tree using the 
\textit{ClusterProfiler} function \textit{buildGOmap}.

Then we conducted gene over-representation analysis with the function 
\textit{compareCluster}, using as universe/background the set of 24011 genes 
detected in at least one good quality leaf RNAseq library. This function 
calculates the hypergeometric test for overrepresented ontology terms in the 
specified gene set and returns raw, and FDR-adjusted p-values. We then manually 
reviewed the combined 1700 significant (\textit{FDR} $<0.05$) overrepresented 
GO term associations for the 6 predictor/regulation combinations, and we 
selected for illustration an \textit{ad hoc} subset with low semantic 
redundancy.

Similarly, We tested for KEGG pathways over representation using the 
\textit{enrichKEGG} function from \textit{compareCluster}, which makes the same 
hypothesis tests on the NCBI REFseq annotation of the B73 NAM assembly. Both 
types of overrepresentation analysis were plotted with the package \textit{DOSE} 
\cite{yu2015}.

\subsection{Filtering of \invfour DEGs by phenotype association}

As our data showed evidence of \invfour accelerating flowering time and 
increasing plant height, we put together a list of candidate genes associated 
with these two phenotypes to tease out which DEGs were likely contributors to 
the observed \invfour effect in these traits. For flowering time, we started 
with the list of 991 genes compiled by \cite{wang2021} and 62 genes from 
\cite{li2023a}.

Then we downloaded the maize data from the GWAS atlas \cite{liu2023} 
(\textit{gwas\_association\_result\_for\_maize.txt.gz}) and selected genes that 
overlapped association SNPs for the 
\href{https://ngdc.cncb.ac.cn/gwas/browse/ontology}{Plant Phenotype and Trait 
Ontology} term ``days to flowering trait" \textit{PPTO:0000155}. For this and 
the following candidate gene list, we considered that a gene overlapped an 
association SNP if the SNP was located within the 5 kb extended range of the 
gene model, i.e. as described in the gff gene annotation $\pm 5$ kb.

The final source of associations for flowering time was the phenotypic 
plasticity study in \cite{tibbs-cortes2024} from which we used 281 genes with 
significant GWAS SNPs in the columns \textit{DTS\_slope}, \textit{DTS\_intcp}, 
\textit{DTA\_slope}, \textit{DTA\_intcp}. For plant height, 27 genes from 
\cite{liu2023}, 1210 genes with GWAS Atlas associations for the term ``plant 
height" \textit{PPTO:0000126}; and 39 genes overlapping phenotypic plasticity 
association SNPs for \textit{PH\_slope} and \textit{PH\_intcp} 
\cite{tibbs-cortes2024}. The final nonredundant list consisted of a total of 
2224 candidate genes for flowering time and 1272 candidates for plant height.

\subsection{Meristem clearing and size quantification}

For vegetative meristem size quantification, maize seedlings were grown at the 
North Carolina State University Phytotron in a Percival Model LT-105 growth 
chamber (conditions: 29.4°C day/23.9°C night, relative humidity 50\%, 16 hours 
light/8 hours dark, light intensity of 412 $\mu$ mol at plant height; soil 
type: 1:1 Sun Gro Propagation Growing Mix [Canadian Sphagnum peat moss 50-65\%, 
vermiculite, dolomitic lime, 0.0001\% silicon dioxide] : cement sand) in 
24-well trays. Seedlings were watered daily in the morning daily and fertilized 
three times per week.

Two-weeks after planting, seedlings were cut at soil level and again 1cm above 
the soil cut. This 1cm tissue cassette of the shoot apex was cut longitudinally 
in the medial plane, in a midrib to margin orientation, by hand with a razor 
blade. Tissue was fixed in ice cold and fresh FAA [50\% EtOH, 35\% milliQ 
water, 10\% formaldehyde (35\%), 5\% glacial acetic acid (v/v)] with a vacuum 
for 15 min. FAA was replaced, and tissues were place overnight at 4°C on a 
rocker.

For clearing, tissues were then removed from FAA and dehydrated through a 
graded ethanol series at room temperature for an hour each with gentle shaking: 
50\%, 70\%, 85\%, 95\% EtOH (v/v)], followed by 1:1 95\% EtOH and Methyl 
Salicylate, and finally, 100\% Methyl Salicylate. Once in 100\% Methyl 
Salicylate, samples were left to shake at room temperature overnight. Each 
shoot apex tissue cassette was placed on a microscope slide and covered with a 
coverslip.

Cleared shoot apices were imaged with differential internal contrast using a 
Leica DM4B microscope equipped with a DMC6200 digital camera. Each shoot apex 
image was measured using ImageJ v2.14.0. The scale was properly set for each 
image. Width was measured as a straight line at 0° anchored from the edge of 
P0. Height was measured from the highest point of the meristem tip to the width 
line at 90°. Surface area was measured using the polygon selection tool by 
taking into consideration the width line and marking the edge of the meristem 
dome.

Data were analyzed using ANOVA and plots were generated in R v.4.3.2 with the 
packages rstatix v.0.7.2, readxl v.1.4.3, ggplot2 v.3.5.1 and ggpubr v.0.6.0.

\begin{figure*}[!ht]
\includegraphics[width=0.95\linewidth]{figs/jmj_cluster.png}
\centering
\caption{\textbf{The \jmjii cluster within \invfour shows lineage-specific expansion in B73.}
\textbf{(A)} Microsynteny of the JUMONJI gene cluster across three \textit{Zea} genomes.
The teosinte TIL18 (\textit{Z. mays} ssp. \textit{mexicana}) and the highland landrace Palomero Toluque\~{n}o (PT) each contain a single \textit{jmj6} ortholog, while B73 has an expanded tandem array of five paralogs (\textit{jmj9}, $\psi$?, \textit{jmj6},
\textit{jmj2}, \textit{jmj4}). 
Purple ribbons indicate syntenic relationships between \textit{jmj} loci; gray ribbons show conserved flanking genes. Coordinates in Mb.
\textbf{(B)} Transcript structure and annotation of the B73 \textit{jmj} cluster.
The NAM v5 annotation consolidates \textit{jmj2}, \textit{jmj6}, and \textit{jmj9} as alternative transcripts of a single locus (\textit{Zm00001eb191790}), while
\textit{jmj4} is annotated separately (\textit{Zm00001eb191820}). 
The putative pseudogene ($\psi$?) retains only a v4 identifier (\textit{Zm00001d051961}).
Heatmap intensity indicates cDNA sequence identity (97--100\%). Bottom track shows genomic positions of individual cluster members.}
\label{fig:jmj_cluster}
\end{figure*}


\section{Discussion}

Our study demonstrates that introgression of the \invfour chromosomal inversion 
from Mexican highland maize into temperate germplasm produces significant 
phenotypic effects on flowering time and plant height, independent of phosphorus 
availability. These findings illuminate potential mechanisms underlying local 
adaptation in maize and identify candidate genes for future functional studies.

The accelerated flowering time in \invfour-carrying plants represents a 
counterintuitive result given the inversion's highland origin, where delayed 
flowering at low elevations is adaptive. However, this pattern is consistent 
with gene-by-environment interactions previously observed for \invfour 
\cite{crow2020}. In temperate environments with extended growing seasons, the 
earlier flowering conferred by \invfour may not be adaptive, explaining its 
rarity in modern temperate cultivars despite potential introgression 
opportunities during maize's northward expansion.

The identification of JUMONJI methyltransferases as strong candidates for 
flowering time regulation provides a mechanistic hypothesis for \invfour's 
effects. These genes show differential copy number between modern maize and 
highland teosinte, suggesting structural variation beyond the inversion itself 
may contribute to phenotypic differences. The perturbation of a cell 
proliferation gene network and the observed elongation of shoot apical 
meristems offer a plausible cellular mechanism linking \invfour genotype to
increased plant height.


Contrary to our initial hypothesis, \invfour showed no detectable effect on 
phosphorus starvation responses despite containing classical PSR genes. This 
suggests that phosphorus availability may not be the primary selective pressure 
maintaining \invfour in highland populations, or that any phosphorus-related 
effects are masked in our experimental conditions. Alternative adaptive 
functions related to temperature, growing season length, or other edaphic 
factors warrant further investigation.

The presence of substantial linkage drag flanking \invfour (24 Mb across eight 
generations of backcrossing) highlights the challenges of dissecting inversion 
effects through introgression. Future work employing larger populations or 
additional recombinant lines could help separate effects of genes within the 
inversion proper from those in flanking regions.

Inv4m
\section{Acknowledgments}

We acknowledge support from [funding agencies]. We thank [collaborators] for 
assistance with field work and laboratory analyses.

\bibliography{Inv4m}

\pagebreak
\onecolumn
\section*{Supplement}

\beginsupplement


\begin{table}[!ht]
\caption[\invfour delimitation and breakpoints]{\textbf{\invfour delimitation and breakpoints.}}
\label{tab:breakpoints}
\resizebox{\columnwidth}{!}{%
\begin{tabular}{@{}llrrrrrrl@{}}
\multicolumn{9}{l}{\textbf{\invfour Limits}} \\ \midrule
 &
   &
  \multicolumn{1}{l}{\textbf{Start}} &
  \multicolumn{1}{l}{\textbf{End}} &
  \multicolumn{1}{l}{\textbf{Length}} &
  \multicolumn{2}{l}{\textbf{Start gene}} &
  \multicolumn{2}{l}{\textbf{End gene}} \\ \cmidrule(l){3-9}
\multicolumn{1}{r}{B73} &
   &
  172883881 &
  188131462 &
  15247580 &
  \multicolumn{2}{r}{\textit{Zm00001eb190470}} &
  \multicolumn{2}{r}{\textit{Zm00001eb194800}} \\
\multicolumn{1}{r}{PT} &
   &
  186925484 &
  173486186 &
  13439298 &
  \multicolumn{2}{r}{\textit{Zm00109aa017629}} &
  \multicolumn{2}{r}{\textit{Zm00109aa018009}} \\
\multicolumn{1}{r}{TIL18} &
   &
  180365316 &
  193570652 &
  13205336 &
  \multicolumn{2}{r}{\textit{Zx00002aa015554}} &
  \multicolumn{2}{r}{\textit{Zx00002aa015905}} \\ \cmidrule(l){3-9}
 &
   &
  \multicolumn{1}{l}{} &
  \multicolumn{1}{l}{} &
  \multicolumn{1}{l}{} &
  \multicolumn{1}{l}{} &
  \multicolumn{1}{l}{} &
  \multicolumn{1}{l}{} &
   \\
\multicolumn{9}{l}{\textbf{\invfour Breakpoints}} \\ \midrule
\textbf{Position} &
   &
  \multicolumn{3}{c}{\textbf{Upstream Segment}} &
  \multicolumn{3}{c}{\textbf{Downstream Segment}} &
  \multicolumn{1}{c}{\textbf{}} \\ \cmidrule(lr){3-8}
 &
   &
  \multicolumn{1}{l}{\textbf{Start}} &
  \multicolumn{1}{l}{\textbf{End}} &
  \multicolumn{1}{l}{\textbf{Span}} &
  \multicolumn{1}{l}{\textbf{Start}} &
  \multicolumn{1}{l}{\textbf{End}} &
  \multicolumn{1}{l}{\textbf{Span}} &
  \textbf{} \\ \cmidrule(lr){3-8}
\multicolumn{1}{r}{B73} &
   &
  \cellcolor[HTML]{FFFFFF}{\color[HTML]{333333} 172561330} &
  \cellcolor[HTML]{FFFFFF}{\color[HTML]{333333} 172883881} &
  322551 &
  188131461 &
  188220418 &
  88957 &
  \multicolumn{1}{r}{} \\
\multicolumn{1}{r}{PT} &
   &
  173369064 &
  \cellcolor[HTML]{FFFFFF}{\color[HTML]{333333} 173486186} &
  117122 &
  186925483 &
  187092654 &
  167171 &
  \multicolumn{1}{r}{} \\
\multicolumn{1}{r}{TIL18} &
   &
  \cellcolor[HTML]{FFFFFF}{\color[HTML]{333333} 180269950} &
  180365316 &
  95366 &
  \cellcolor[HTML]{FFFFFF}{\color[HTML]{333333} 193570651} &
  193734606 &
  163955 &
  \multicolumn{1}{r}{} \\ \cmidrule(lr){3-8}
 &
   &
  \multicolumn{1}{l}{} &
  \multicolumn{1}{l}{} &
  \multicolumn{1}{l}{} &
  \multicolumn{1}{l}{} &
  \multicolumn{1}{l}{} &
  \multicolumn{1}{l}{} &
   \\
\textbf{Genes} &
  \multicolumn{4}{c}{\textbf{Upstream bounding}} &
  \multicolumn{4}{c}{\textbf{Downstream bounding}} \\ \cmidrule(l){2-9}
\multicolumn{1}{r}{B73} &
  \multicolumn{2}{r}{\textit{Zm00001eb190450}} &
  \multicolumn{2}{r}{\textit{Zm00001eb190470}} &
  \multicolumn{2}{r}{\cellcolor[HTML]{FFFFFF}\textit{Zm00109aa017627}} &
  \multicolumn{2}{r}{\cellcolor[HTML]{FFFFFF}\textit{Zm00001eb194820}} \\
\multicolumn{1}{r}{PT} &
  \multicolumn{2}{r}{\textit{Zm00109aa017627}} &
  \multicolumn{2}{r}{\cellcolor[HTML]{FFFFFF}\textit{Zm00109aa017629}} &
  \multicolumn{2}{r}{\cellcolor[HTML]{FFFFFF}\textit{Zx00002aa015905}} &
  \multicolumn{2}{r}{\textit{Zm00109aa017629}} \\
\multicolumn{1}{r}{TIL18} &
  \multicolumn{2}{r}{\textit{Zx00002aa015905}} &
  \multicolumn{2}{r}{\textit{Zx00002aa015906}} &
  \multicolumn{2}{r}{\textit{Zx00002aa015906}} &
  \multicolumn{2}{r}{\textit{Zx00002aa015554}} \\ \bottomrule
\end{tabular}%
}
\end{table}

\begin{table}[!ht]
\caption[\invfour breakpoint knob repeats]{\textbf{\invfour breakpoint knob repeats.} Distribution of 180 bp knob repeats in the breakpoint segments and internal to the inversion.}
\label{tab:knob_repeats}
\resizebox{\columnwidth}{!}{%
\begin{tabular}{@{}llrrrrrrr@{}}
\multicolumn{9}{l}{\textbf{Upstream Breakpoint Segment}} \\ \midrule
 &
   &
  \multicolumn{1}{l}{\textbf{\begin{tabular}[c]{@{}l@{}}Repeat \\ count\end{tabular}}} &
  \multicolumn{1}{l}{\textbf{\begin{tabular}[c]{@{}l@{}}Repeat\\ match\\ length\end{tabular}}} &
  \multicolumn{1}{l}{\textbf{Start}} &
  \multicolumn{1}{l}{\textbf{End}} &
  \multicolumn{1}{l}{\textbf{Span}} &
  \multicolumn{1}{l}{\textbf{\begin{tabular}[c]{@{}l@{}}\% Knob\\ Span\end{tabular}}} &
  \textbf{\% Knob} \\ \cmidrule(l){3-9}
\multicolumn{1}{r}{B73} &
   &
  352 &
  59580 &
  172561959 &
  172882309 &
  320350 &
  99\% &
  \multicolumn{1}{r}{19\%} \\
\multicolumn{1}{r}{PT} &
   &
  44 &
  7638 &
  173370220 &
  173392066 &
  21846 &
  19\% &
  \multicolumn{1}{r}{35\%} \\
\multicolumn{1}{r}{TIL18} &
   &
  44 &
  7638 &
  180271106 &
  180292951 &
  21845 &
  19\% &
  \multicolumn{1}{r}{35\%} \\ \midrule
 &
   &
  \multicolumn{1}{l}{} &
  \multicolumn{1}{l}{} &
  \multicolumn{1}{l}{} &
  \multicolumn{1}{l}{} &
  \multicolumn{1}{l}{} &
  \multicolumn{1}{l}{} &
   \\
\multicolumn{9}{l}{\textbf{\invfour Internal Knob Repeat}} \\ \midrule
 &
   &
  \multicolumn{1}{l}{\textbf{\begin{tabular}[c]{@{}l@{}}Repeat \\ count\end{tabular}}} &
  \multicolumn{1}{l}{\textbf{\begin{tabular}[c]{@{}l@{}}Repeat \\ match\\ length\end{tabular}}} &
  \multicolumn{1}{l}{\textbf{Start}} &
  \multicolumn{1}{l}{\textbf{End}} &
  \multicolumn{1}{l}{\textbf{Span}} &
  \multicolumn{1}{l}{} &
   \\ \cmidrule(lr){3-7}
\multicolumn{1}{r}{B73} &
   &
  650 &
  111841 &
  184152738 &
  184525111 &
  372373 &
  \multicolumn{1}{l}{} &
   \\
\multicolumn{1}{r}{PT} &
   &
  354 &
  61103 &
  177250166 &
  177516302 &
  266136 &
  \multicolumn{1}{l}{} &
   \\
\multicolumn{1}{r}{TIL18} &
   &
  406 &
  69913 &
  183816987 &
  184072303 &
  255316 &
  \multicolumn{1}{l}{} &
   \\ \midrule
 &
   &
  \multicolumn{1}{l}{} &
  \multicolumn{1}{l}{} &
  \multicolumn{1}{l}{} &
  \multicolumn{1}{l}{} &
  \multicolumn{1}{l}{} &
  \multicolumn{1}{l}{} &
   \\
\multicolumn{9}{l}{\textbf{Downstream Breakpoint Segment}} \\ \midrule
 &
   &
  \multicolumn{1}{l}{\textbf{\begin{tabular}[c]{@{}l@{}}Repeat\\ count\end{tabular}}} &
  \multicolumn{1}{l}{\textbf{\begin{tabular}[c]{@{}l@{}}Repeat\\ match\\ length\end{tabular}}} &
  \multicolumn{1}{l}{\textbf{Start}} &
  \multicolumn{1}{l}{\textbf{End}} &
  \multicolumn{1}{l}{\textbf{Span}} &
  \multicolumn{1}{l}{\textbf{\begin{tabular}[c]{@{}l@{}}\% Knob \\ Span\end{tabular}}} &
  \textbf{\% Knob} \\ \cmidrule(l){3-9}
\multicolumn{1}{r}{B73} &
   &
  26 &
  4367 &
  188149304 &
  188219984 &
  70680 &
  79\% &
  \multicolumn{1}{r}{6\%} \\
\multicolumn{1}{r}{PT} &
   &
  130 &
  21466 &
  186927027 &
  187051351 &
  124324 &
  74\% &
  \multicolumn{1}{r}{17\%} \\
\multicolumn{1}{r}{TIL18} &
   &
  154 &
  25847 &
  193572198 &
  193671081 &
  98883 &
  59\% &
  \multicolumn{1}{r}{26\%} \\ \bottomrule
\end{tabular}%
}
\end{table}


\begin{figure}[!ht]
\includegraphics[width=\linewidth]{figs/SNP_distribution.png}
\caption{\textbf{Chromosome Distribution of the SNPs in the RNAseq samples, 
their correlation with \invfour, and genotype run length as percentage of the 
genome.} 
\textbf{(A)} Chromosome 4 contains 44\% of all genotyped SNPs and 98\% 
of \invfour correlated SNPs (n=13).
\textbf{(B)} Manhattan plot for the significance (\textit{t-test}) of the Pearson correlation of 19861 genotyped SNPs with PZE04175660223. Red line: \textit{FDR} $=0.05$ threshold. 
\textbf{(C)} Pearson correlation ($r$) of the each SNP with the \invfour tagging SNP PZE04175660223, open circles: non-significant SNPs.
SNPs with significant correlation with \invfour (full circles) are divergent highland introgressions.
\textbf{(D)} NIL genotype run length as percentage of the genome. Each point represents a NIL, \invfour plants are tagged by the highland allele of PZE04175660223, CTRL plants by the reference (B73) allele, dashed line is the overall mean for the 13 NILs.}
\label{fig::SNPdistro}
\end{figure}


\begin{table}[!ht]
\caption{\textbf{Effect of \invfour on gene expression.} Selected DEGs from the plant-blocking limma model (FDR $< 0.05$, $|log_2 FC| > 1.5$). Bottom row shows the only significant Leaf$\times$\invfour interaction.}
\label{tab:gene_effects}
\resizebox{\textwidth}{!}{%
\begin{tabular}{@{}rrrrrr@{}}
\toprule
\textbf{Predictor} & \textbf{Regulation} & \textbf{Locus} & \textbf{-log10(\textit{FDR})} & \textbf{logFC} & \textbf{Name}                                   \\ \midrule
\invfour & Down & jmj2      & 21.9 & -3.49 & JUMONJI-transcription factor 2              \\
\invfour & Down & jmj4      & 19.7 & -2.84 & JUMONJI-transcription factor 4              \\
\invfour & Down & metrs1    & 17.5 & -2.86 & Methionyl-tRNA synthetase                   \\
\invfour & Down & uba2      & 17.5 & -1.98 & SUMO-activating enzyme subunit              \\
\invfour & Down & ccta      & 15.8 & -1.69 & CCT-alpha                                   \\
\invfour & Down & arf1      & 15.8 & -2.78 & ADP-ribosylation factor                     \\
\invfour & Down & hsp70-17  & 14.2 & -6.48 & Heat shock 70 kDa protein 17                \\
\invfour & Down & Zm..0240  & 14.2 & -9.16 & ---                                         \\
\invfour & Down & ropgef14  & 13.6 & -2.98 & Rop guanine nucleotide exchange factor 14   \\
\invfour & Down & trxl2     & 13.2 & -6.10 & Thioredoxin-like 3-1, chloroplastic         \\
\invfour & Down & ank       & 13.0 & -1.83 & Ankyrin repeat-containing protein           \\
\rowcolor[HTML]{EFEFEF}
\invfour & Up   & ychf      & 16.9 & 6.50  & ---                                         \\
\rowcolor[HTML]{EFEFEF}
\invfour & Up   & sec6      & 16.5 & 2.87  & Exocyst complex component Sec6              \\
\rowcolor[HTML]{EFEFEF}
\invfour & Up   & zfrvt     & 15.6 & 1.97  & ---                                         \\
\rowcolor[HTML]{EFEFEF}
\invfour & Up   & pig3      & 15.0 & 1.97  & Quinone oxidoreductase PIG3                 \\
\rowcolor[HTML]{EFEFEF}
\invfour & Up   & etfa      & 14.0 & 4.46  & Electron transfer flavoprotein alpha        \\
\rowcolor[HTML]{EFEFEF}
\invfour & Up   & gbp1      & 13.7 & 8.96  & GTP-binding protein-related                 \\
\rowcolor[HTML]{EFEFEF}
\invfour & Up   & Zm..5370  & 13.6 & 1.63  & ---                                         \\
\rowcolor[HTML]{EFEFEF}
\invfour & Up   & kan1      & 13.6 & 1.68  & KANADI1                                     \\
\rowcolor[HTML]{EFEFEF}
\invfour & Up   & aldh2     & 13.6 & 2.70  & aldehyde dehydrogenase2                     \\
\rowcolor[HTML]{EFEFEF}
\invfour & Up   & jmj21     & 13.5 & 2.28  & JUMONJI-transcription factor 21             \\
\rowcolor[HTML]{EFEFEF}
\invfour & Up   & Zm..9580  & 13.2 & 7.38  & ---                                         \\
\rowcolor[HTML]{EFEFEF}
\invfour & Up   & mybsant   & 12.4 & 6.96  & ---                                         \\
\rowcolor[HTML]{EFEFEF}
\invfour & Up   & engd2     & 12.3 & 6.14  & ---                                         \\
\rowcolor[HTML]{EFEFEF}
\invfour & Up   & gpm458    & 11.2 & 10.47 & Oil body-associated protein like            \\ \midrule
Leaf$\times$\invfour & Down & sec6 & 3.1 & -0.55 & Exocyst complex component Sec6              \\ \bottomrule
\end{tabular}%
}
\end{table}


\end{document}