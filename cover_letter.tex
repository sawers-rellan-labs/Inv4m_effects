\documentclass[11pt]{article}
\usepackage[utf8]{inputenc}
\usepackage[T1]{fontenc}
\usepackage[letterpaper, margin=0.75in]{geometry}
\usepackage{hyperref}
\usepackage{enumitem}
\usepackage{graphicx} % Required for inserting images

% Set paragraph formatting for a block letter style
\setlength{\parindent}{0pt}
\setlength{\parskip}{1em}

\begin{document}

% --- Logo Placement ---
% Using flushright as requested in the previous interaction
\begin{flushright}
    \includegraphics[width=0.4\textwidth]{logo.png}
\end{flushright}
\vspace{0.5em}

% --- Date ---
\today

% --- Recipient Block ---
\textbf{Dr. Nonia Pariente} \\
\textit{PLOS Biology}

\vspace{1em}

% --- Subject Line ---
\textbf{Subject:} Submission of manuscript ``Introgression of Mexican highland chromosomal inversion \textit{Inv4m} into temperate maize accelerates flowering, promotes growth, and modulates a cell proliferation gene network''

\vspace{1em}

Dear Editors,

We are pleased to submit our manuscript, titled ``\textbf{Introgression of Mexican highland chromosomal inversion \textit{Inv4m} into temperate maize accelerates flowering, promotes growth, and modulates a cell proliferation gene network},'' for consideration for publication in \textit{PLOS Biology}.

Chromosomal inversions are widely recognized as engines of local adaptation because they suppress recombination, locking together combinations of alleles into "supergenes." While their presence is frequently associated with fitness in divergent environments, the specific physiological and molecular mechanisms by which they confer this fitness often remain a "black box." Our study pierces this opacity. Focusing on \textit{Inv4m}, a 15 Mb inversion critical for maize adaptation to the Mexican highlands, we provide a rare mechanistic dissection of how a large structural variant reshapes plant development to ensure survival.

In this study, we utilize Near Isogenic Lines (NILs) to isolate the effects of the highland \textit{Inv4m} haplotype within a temperate B73 genetic background. By combining extensive field phenotyping with transcriptomic network analysis and microscopy, we identify the developmental machinery modulated by this inversion.

Our key findings include:
\begin{enumerate}[leftmargin=*, label=\textbf{\arabic*.}]
    \item \textbf{Developmental Acceleration:} Contrary to the expectation that highland adaptation involves slower growth, we found that \textit{Inv4m} significantly accelerates flowering and increases plant height in temperate environments. Crucially, these traits are robust and independent of soil phosphorus availability.
    
    \item \textbf{Cellular Mechanism:} We link these macroscopic phenotypes to specific cellular changes, demonstrating that \textit{Inv4m} plants possess significantly elongated shoot apical meristems (SAMs).
    
    \item \textbf{Genetic Candidates:} Transcriptomic analysis revealed the perturbation of a cell proliferation network and the specific downregulation of \textit{JUMONJI} methyltransferases. These genes show copy number variation between modern maize and highland teosinte, suggesting they are key drivers of the inversion's phenotypic effects.
\end{enumerate}

\textbf{Co-submission Statement:}
We would like to inform you that we are simultaneously submitting a companion manuscript to \textbf{\textit{PLOS Genetics}}. In that manuscript, we utilized the same NIL population to test the hypothesis that \textit{Inv4m} drives adaptation via phosphorus use efficiency. That study proves that the inversion does \textit{not} alter the metabolic response to phosphorus starvation, a negative result that was critical in directing our focus toward the developmental timing mechanisms detailed here. 

Together, these two manuscripts provide a comprehensive dissection of a major adaptive structural variant. While the \textit{PLOS Genetics} paper defines what \textit{Inv4m} does \textit{not} do (metabolic adaptation), this \textit{PLOS Biology} manuscript defines what it \textit{does} do: it fundamentally alters the plant's developmental clock and meristematic architecture to fit constrained growing seasons.

We believe this work is of broad interest to the \textit{PLOS Biology} readership because it moves beyond the simple association of structural variants with environmental variables. Instead, it challenges the assumption that adaptation to harsh environments is always driven by stress tolerance (e.g., phosphorus efficiency), revealing that constitutive shifts in developmental timing and cell proliferation are the primary evolutionary levers pulled by this inversion.

This manuscript has not been published and is not under consideration for publication elsewhere. All authors have read and approved the manuscript.

Thank you for your time and consideration.

Sincerely,

\vspace{2em}

% --- Signature Block ---
\textbf{Fausto Rodr\'iguez-Zapata, Rub\'en Rell\'an-\'Alvarez} \\
Department of Molecular and Structural Biochemistry \\
N.C. Plant Sciences Initiative \\
North Carolina State University, Raleigh, NC, USA \\
\href{mailto:frodrig4@ncsu.edu}{frodrig4@ncsu.edu}, \href{mailto:rrellan@ncsu.edu}{rrellan@ncsu.edu}

\end{document}